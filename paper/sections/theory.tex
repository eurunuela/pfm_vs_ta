% !TEX root = ../main.tex

\section{Theory}

% \begin{itemize}
%     \item What is deconvolution and different formulations presented as a review.
%     \item Analysis vs synthesis
%     \begin{itemize}
%         \item TA paper but without the spatial regularization
%         \item PFM paper
%         \item In Gitelman it's an \(\mathbf{H}\) multiplied by a Fourier term.
%     \end{itemize}
%     \item Spikes and block models
% \end{itemize}

\todo[inline]{I think we need a short paragraph here introducing: 1) our notations (vectors, matrices, discrete, continuous); 2) basic math definitions, in particular for norms.}

Conventional general linear model (GLM) analysis puts forward a number of regressors incorporating knowledge about the paradigm or behavior. For instance, the timing of epochs for a certain condition can be modeled as an indicator function $p(t)$, convolved with the HRF $h(t)$, and sampled at TR resolution (\citealt{friston1994analysis, friston1998event, boynton1996linear, cohen1997parametric}):\todo{Rather some Friston paper(s) here}
$$
   p(t) \rightarrow p*h(t) \rightarrow x[k] = p*h(k\cdot\text{TR}).
$$
The vector $\mathbf{x}=[x[k]]_{k=1,\ldots,N}$ then constitutes the hypothetical response, and several of them can be stacked as the columns of the design matrix $\mathbf{X}=[\mathbf{x}_1 \ldots \mathbf{x}_L]$, leading to the celebrated GLM: 
\begin{equation}
    \label{eq:glm}
    \mathbf{y} = \mathbf{X \beta} + \mathbf{e},
\end{equation}
where the empirical timecourse $\mathbf{y}$ is explained by a linear combination of the regressors in $\mathbf{X}$ weighted by the parameter weights in $\mathbf{\beta}$ and corrupted by additive noise $\mathbf{e}$. Under independent and identically distributed Gaussian assumptions of the latter, the maximum likehood estimate of the parameter weights reverts to the ordinary least-squares estimator; i.e., minimizing the residual sum of squares between the fitted model and measurements. The number of regressors $L$ is typically much less than the number of measurements $N$, and thus the regression problem is over-determined and does no require additional constraints or assumptions.

In the deconvolution approach, no prior knowledge is taken into account, and the purpose is to estimate the deconvolved activity-inducing signal $\mathbf{s}$ from the measurements $\mathbf{y}$, which can be formulated as the signal model
\begin{equation}
    \label{eq:deconvolution}
    \mathbf{y} = \mathbf{Hs} + \mathbf{e},
\end{equation}
where $\mathbf{H}$ is an $N\times N$ Toeplitz matrix that represents the discrete convolution with the HRF, and $\mathbf{s}$ a length-$N$ vector with the unknown activity-inducing signal. Despite the apparent similarity with the GLM equation, there are two important differences. First, the multiplication with the design matrix of the GLM is an expansion as a weighted linear combination of its columns, while the multiplication with the HRF matrix represents a convolution with its shifted rows. Second, determining $\mathbf{s}$ is an ill-posed problem given the nature of the HRF; e.g., as can be seen intuitively, the rows of $\mathbf{H}$ are highly correlated due to large overlap between shifted HRFs (see Figure~\ref{fig:hrf_diff}B), thus introducing ambiguity and instability in the estimates of $\mathbf{s}$. Therefore, additional assumptions under the form of regularization are needed. 

%%%%%%%%%%%%%%%%%%%%%%%%%%%%%%%%%%%%%%%%%%%%%%%%%%%%%%%%%%%%%%%%%%%%%%%%
% Paradigm Free Mapping
%%%%%%%%%%%%%%%%%%%%%%%%%%%%%%%%%%%%%%%%%%%%%%%%%%%%%%%%%%%%%%%%%%%%%%%%

\subsection{Paradigm-free mapping}
The first approach for deconvolution is based on the forward model formulation of Eq.~(\ref{eq:deconvolution}); i.e., the to-be-recovered activity-inducing signal $\hat{\mathbf{s}}$ is convolved by the HRF and compared against the measured timecourse in the least-squares sense: 
\begin{equation}
    \label{eq:pfm}
    \hat{\mathbf{s}} = \arg \min_{\mathbf{s}} \frac{1}{2} \| \mathbf{y} - \mathbf{Hs} \|_2^2 + \Omega(\mathbf{s}).
\end{equation}
The first term quantifies data fitness, which can be justified as the log-likelihood term derived from Gaussian noise assumptions, while the second term \(\Omega(\mathbf{s})\) brings in regularization and be interpreted as a prior on the activity-inducing signal. In classical Wiener deconvolution, the (squared) $\ell_2$-norm of $\mathbf{s}$ is imposed (i.e., $\Omega(\mathbf{s})=\lambda \left\| \mathbf{s}\right\|_2^2$), which introduces a trade-off, controlled by the regularization parameter $\lambda$, between data fit and ``energy'' of the solution. In paradigm-free mapping (PFM), the formulation of Eq.~(\ref{eq:pfm}) was considered equivalently as fitting the measurements using the atoms of a dictionary (columns of $\mathbf{H}$) with corresponding weights (entries of $\mathbf{s}$) (\citealt{gaudes2011DetectionCharacterizationSingletrial,caballerogaudes2013ParadigmFreeMapping,urunuela2020StabilityBasedSparseParadigm}). 
In addition, sparsity-pursuing regularization was introduced on $\mathbf{s}$, which in a strict way reverts to choosing \(\Omega(\mathbf{s})=\lambda \| \mathbf{s} \|_0\) and solving the optimization problem (\citealt{bruckstein2009SparseSolutionsSystems}). However, due to the convolution model\todo{Even without would be the case, right?} defined in~\eqref{eq:pfm}, finding the optimal solution to the problem demands an exhaustive search across all possible combinations of the columns of \(\mathbf{H}\). Hence, a  pragmatic solution is to solve the convex-relaxed optimization problem for the \(l_1\)-norm, commonly known as the LASSO (\citealt{tibshirani1996RegressionShrinkageSelection}): 
\begin{equation}
    \label{eq:pfm_spike}
    \hat{\mathbf{s}} = \arg \min_{\mathbf{s}} \frac{1}{2} \| \mathbf{y} - \mathbf{Hs} \|_2^2 + \lambda \| \mathbf{s} \|_1,
\end{equation}
which provides fast convergence to a global solution. From the neuronal perspective, imposing sparsity on the activity-inducing signal expresses that activity, at the fMRI timescale of seconds, is supposed to be short and typically can be explained by a small number of large entries. 



%%%%%%%%%%%%%%%%%%%%%%%%%%%%%%%%%%%%%%%%%%%%%%%%%%%%%%%%%%%%%%%%%%%%%%%%
% Total Activation
%%%%%%%%%%%%%%%%%%%%%%%%%%%%%%%%%%%%%%%%%%%%%%%%%%%%%%%%%%%%%%%%%%%%%%%%

\subsection{Total activation}
Alternatively,  deconvolution can be formulated as a denoising problem where the signal to be recovered is directly fitting the measurements and at the same time satisfying some suitable regularization, which leads to
\begin{equation}
    \hat{\mathbf{x}} = \arg \min_{\mathbf{x}} \frac{1}{2} \| \mathbf{y} - \mathbf{x} \|_2^2 + \Omega(\mathbf{x}).
\end{equation}
Well-known regularized regression techniques such as ridge regression (i.e., $\Omega(\mathbf{x})=\lambda\|\mathbf{x}\|_2^2$) and elastic net (i.e., $\Omega(\mathbf{x})=\lambda_1\|\mathbf{x}\|_2^2 + \lambda_2\|\mathbf{x}\|_1$) fall under this category [REF]. One other powerful regularizer is total variation (TV), which is the $\ell_1$-norm of the derivative, $\Omega(\mathbf{x})=\lambda \|\mathbf{Dx}\|_1$, and favors recovery of piecewise-constant signals [REF]. The approach of generalized TV introduces an additional differential operator $\mathbf{D_H}$ in the regularizer that can be tailored as the inverse operator of a linear system~(\citealt{karahanoglu2011SignalProcessingApproacha}), that is, $\Omega(\mathbf{x})=\lambda \|\mathbf{D D_H x}\|_1$. In the context of hemodynamic deconvolution, total activation is proposed for which the discrete operator $\mathbf{D_H}$ is derived from the inverse of the continuous-domain linearized balloon-windkessel model. Exchanging the poles and zeros of the latter's linear-system characterization leads to a differential operator of the form 
\begin{equation}
    D_H\ = \prod_{i=1}^{M_1} (D-\alpha_i I) (\prod_{j=1}^{M_2} (D - \gamma_j I))^{-1},
\end{equation}
where \(D\) is the derivative operator, \(\alpha_i\) the zeros, and \(\gamma_j\) the poles. The interested reader is referred to (\citealt{karahanoglu2013TotalActivationFMRI}) for a detailed description. 

Therefore, the solution of the total-activation problem 
\begin{equation}
    \hat{\mathbf{x}} = \arg \min_{\mathbf{x}} \frac{1}{2} \| \mathbf{y} - \mathbf{x} \|_2^2 + \lambda \| \mathbf{D D_H x} \|_1
\end{equation}
will render the activity-related signal $\mathbf{x}$ for which the activity-inducing signal $\mathbf{s}=\mathbf{D_H x}$ and so-called innovation signal $\mathbf{u}=\mathbf{Ds}$ will also be available, as they are required for the regularization.  

\subsection{Unifying both perspectives}
\todo[inline]{I would go for a schematic block-flowchart that goes between activity-related/inducing/innovation signals with between each step the two operators; i.e., $\mathbf{H}$ and $\mathbf{D_H}$ for the first, $\mathbf{L}$ and $\mathbf{D}$ for the second.}

PFM and TA are based on the synthesis- and analysis-based formulation of the deconvolution problem, respectively. In the first case, the recovered deconvolved signal is synthesized to be matched to the measurements, while in the second case, the recovered signal is directly matched to the measurements but needs to satisfy its analysis in terms of deconvolution. This also corresponds to using the forward or backward model of the hemodynamic system, respectively. Both approaches can be made equivalent. First, TA can be made equivalent to PFM by removing the derivative operator of TA's regularizer. It can be readily verified that replacing in that case $\mathbf{x}=\mathbf{Hs}$ leads to identical equations. Second, PFM can also be made equivalent to TA, by considering the modified forward model
$$
\mathbf{y} = \mathbf{H L u} + \mathbf{e},
$$
where the activity-inducing signal $\mathbf{s}$ is rewritten in terms of the innovation signal $\mathbf{u}$; i.e., the derivative $\mathbf{Ds}=\mathbf{u}$ of $\mathbf{s}$ (\citealt{cherkaoui2019SparsitybasedBlindDeconvolution,urunuela2020StabilityBasedSparseParadigm}). This way, PFM will solve for the innovation signal $\mathbf{u}$: 
\begin{equation}
    \label{eq:pfm_block}
    \hat{\mathbf{u}} = \arg \min_{\mathbf{u}} \frac{1}{2} \| \mathbf{y} - \mathbf{HLu} \|_2^2 + \lambda \| \mathbf{u} \|_1,
\end{equation}
and becomes equivalent to TA by replacing $\mathbf{u}=\mathbf{D D_H x}$. 

This work evaluates the core of the two techniques, i.e., the regularized least-squares problem with temporal regularization, which corresponds to the generalized total-variation operator in Total Activation. Therefore, we do not study the impact of spatial constraints, as we assume that spatial regularization terms should perform identically on both methods.

\subsection{Algorithms}
\todo[inline]{I think we need a subsection here that gives a brief overview of the different computational approaches to solve PFM and TA. In particular, I think that for the first Figure mentioned above, we could have a second panel (b) in which we put a flowchart of LASSO and FISTA to illustrate the main idea.}

\subsection{Parameter selection}
\label{sec:regparam}

The correct selection of the regularization parameter \(\lambda\) is a critical decision for the accurate performance of deconvolution methods. Even though many techniques have been proposed in the literature, the optimal strategy that selects \(\lambda\) is yet to be found. Algorithms like least angle regression (LARS) (\citealt{efron2004LeastAngleRegression}) provide all the possible solutions to the optimization problem and their corresponding value of \(\lambda\), i.e., the regularization path, but don't provide the optimal solution. Therefore, strategies that exploit the regularization path can provide a selection of \(\lambda\) that is close to the optimal. For instance, in Paradigm Free Mapping, the optimal result is given by the Bayesian Information Criterion (BIC) (\citealt{schwarz1978EstimatingDimensionModel}), i.e., the regularization path solution with the minimum BIC is considered optimal. Another approach could be to update the regularization parameter \(\lambda\) on every iteration \(n\) like Total Activation does, so that the residuals converge to a previously estimated noise level of the data fit \(\tilde{\sigma}\). The pre-estimated noise level is calculated from the median absolute deviation of fine-scale wavelet coefficients (Daubechies, order 3) (\citealt{karahanoglu2013TotalActivationFMRI}):
\begin{equation}
    \lambda^{n+1} = \frac{N \tilde{\sigma}}{\frac{1}{2} \| \mathbf{y} - \mathbf{x}^n \|_F^2} \lambda^n.
\label{eq:std}
\end{equation}

Here, we compare the performance of the two deconvolution algorithms with both selection criteria and in terms of the estimation of the activity-inducing signal \(\mathbf{s}\) using the \textit{spike model} in~\eqref{eq:pfm_spike} and the innovation signal \(\mathbf{u}\) using the \textit{block model} in~\eqref{eq:pfm_block}.