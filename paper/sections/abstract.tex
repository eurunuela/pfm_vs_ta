Deconvolution of the hemodynamic response is an important step to access short timescales of brain activity recorded by functional magnetic resonance imaging (fMRI). Albeit conventional deconvolution algorithms have been around for a long time (e.g., Wiener deconvolution), recent state-of-the-art methods based on sparsity-pursuing regularization are attracting increasing interest to investigate brain dynamics and connectivity. This technical note revisits the main concepts underlying two main methods, Paradigm Free Mapping and Total Activation, in the most accessible way. Despite their apparent differences, these methods are theoretically equivalent as they represent the synthesis and analysis sides of the same problem. We demonstrate this equivalence in practice with their best-available implementations using both simulations, with different signal-to-noise ratios, and experimental data of motor task and resting-state fMRI. We evaluate the parameter settings that lead to equivalent results, and benchmark the computational speed of both algorithms. This note is useful for practitioners interested in having a better understanding of state-of-the-art hemodynamic deconvolution, and who want to make use of them in the most efficient implementation.