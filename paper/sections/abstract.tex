Deconvolution of the hemodynamic response is an important step to access short
timescales of brain activity recorded by functional magnetic resonance imaging
(fMRI). Albeit conventional deconvolution algorithms have been around for a long
time (e.g., Wiener deconvolution), recent state-of-the-art methods based on
sparsity-pursuing regularization are attracting increasing interest to
investigate brain dynamics and connectivity with fMRI. This technical note
revisits the main concepts underlying two main methods, Paradigm Free Mapping
and Total Activation, in the most accessible way. Despite their apparent
differences in the formulation, these methods are theoretically equivalent as
they represent the synthesis and analysis sides of the same problem,
respectively. We demonstrate this equivalence in practice with their
best-available implementations using both simulations, with different
signal-to-noise ratios, and experimental fMRI data acquired during a motor task
and resting-state. We evaluate the parameter settings that lead to equivalent
results, and showcase the potential of these algorithms compared to other common
approaches. This note is useful for practitioners interested in gaining a better
understanding of state-of-the-art hemodynamic deconvolution, and aims to answer
questions that practitioners often have regarding the differences between the
two methods.