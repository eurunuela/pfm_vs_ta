% !TEX root = ../main.tex

\section{Discussion}

% New command for table
\newcommand\pro{\item[$+$]}
\newcommand\con{\item[$-$]}

% \begin{itemize}
%     \item Pros and cons of each formulation: analysis vs synthesis
%     \item Link with other approaches
%     \item Finish with conclusions and a moving forward
%     \begin{itemize}
%         \item We have to refine the deconvolution
%         \item HRF variability there are three: conference proceeding by Philippe~\citealt{badillo2013group}, ISBI 2012 by César~\citealt{gaudes2012structured}, and Farouj with a different formulation. Say conceptual differences among those.
%         \item Mention stability-selection~\citealt{meinshausen2010stability}
%         \item Debiasing
%         \item Connected to debiasing other deconvolution algorithms that are based on a norm lower than 1.
%     \end{itemize}
% \end{itemize}

\begin{table}
    \centering
    \begin{tabular}{ m{4cm} m{4cm} }
        Paradigm Free Mapping & Total Activation \\
        \toprule
        \begin{itemize}
            \pro Its formulation can be extended straightforwardly for deconvolution of multiple signals with a common neuronal-related signal, e.g., for multi-echo formulations \citealt{caballero-gaudes2019DeconvolutionAlgorithmMultiecho}.
            \pro The model can implement any HRF shape very easily since it only requires the coefficients at the required temporal resolution.
            \con 
        \end{itemize} &
        \begin{itemize}
            \pro Both the spike and block models solve the regularization problem with the same HRF.
            \con 
        \end{itemize} \\
    \end{tabular}
    \caption{Advantages (+) and disadvantages (-) of Paradigm Free Mapping and Total Activation with respect to each other.}
    \label{tab:proscons}
\end{table}

\todo[inline]{Should be expanded to provide a clear message.}
This work demonstrates that PFM and TA yield practically identical results when the same HRF model and equivalent regularization parameter are employed, demonstrating that synthesis and analysis models are equivalent for temporal fMRI deconvolution. Thus, previously observed differences in performance must be due to differences in usage options. With the equivalence in the temporal deconvolution demonstrated, it is reasonable to assume that additional regularization terms in the spatial or temporal domains would not modify this equivalence when convex operators are employed; e.g., when the regularization problem can be solved by means of the Fast Iterative Shrinkage-Thresholding Algorithm (FISTA) (\citealt{Beck2009FastIterativeShrinkage}) or the Generalized Forward-Backward Splitting (\citealt{Raguet2013GeneralizedForwardBackward}) techniques\todo{Again, move up}. Our findings are in line with the equivalence of analysis and synthesis methods in under-determined cases (\(N \leq V\))\todo{Ola, what is $V$? :-) Move up in Section on analysis-synthesis equivalence} as demonstrated in (\citealt{Elad2007Analysisversussynthesis}).

Taking into account the advantages and disatvantages of the presented techniques shown in Table~\ref{tab:proscons}, future work will improve and extend deconvolution methods for fMRI. For instance, the appropriate formulation depending on data acquisition (i.e., single-echo vs multi-echo) could be studied and compared with existing methods (\citealt{CaballeroGaudes2019deconvolutionalgorithmmulti}), or formulations that account for HRF variability could be investigated too (\citealt{Badillo2013Grouplevelimpacts,Gaudes2012Structuredsparsedeconvolution,Farouj2019BoldSignalDeconvolution}). Furthermore, robust methods to select the regularization parameter (\citealt{Urunuela2020StabilityBasedSparse,Meinshausen2010Stabilityselection}) and other potential \(\ell_{p,q}\)-norm regularization terms (e.g., \(p < 1\)) or debiasing approaches could be explored.

\todo[inline]{Folks, I think we need to polish the message for the reader further. If the methods are equivalent, either the reader has learned something about different types of optimization (but this is not yet well described in the paper), or there is a clear practical take-home message.}



A series of recent studies have attempted to understand neural processes by studying the interactions between BOLD responses without estimating the underlying neuronal activity. For instance, co-activation patterns have been used to replicate seed correlation-based resting-state functional networks with a small portion of the data (\citealt{Liu2013Timevaryingfunctional,Liu2013Decompositionspontaneousbrain,Liu2018Coactivationpatterns,Majeed2009Spatiotemporaldynamicslow,Majeed2011Spatiotemporaldynamicslow,cifre2020revisiting,Cifre2020Furtherresultswhy,Zhang2020relationshipBOLDneural}). Likewise, the dynamics of functional connectivity have recently been investigated with the use of co-fluctuations and edge-centric techniques on tasks (\citealt{Faskowitz2021edgecentricmodel}), resting-state (\citealt{Esfahlani2020Highamplitudecofluctuations}) and naturalistic paradigms (\citealt{Faskowitz2020Edgecentricfunctional,Betzel2020Temporalfluctuationsbrains}). %Methods based on the multiplication of temporal derivatives have also been presented for the estimation of dynamic functional connectivity on task fMRI data (\citealt{shine2015estimation,shine2016dynamics}).