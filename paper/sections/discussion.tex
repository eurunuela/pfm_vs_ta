\section{Discussion}

% New command for table
\newcommand\pro{\item[$+$]}
\newcommand\con{\item[$-$]}

\begin{itemize}
    \item Pros and cons of each formulation: analysis vs synthesis
    \item Link with other approaches
    \item Finish with conclusions and a moving forward
    \begin{itemize}
        \item We have to refine the deconvolution
        \item HRF variability there are three: conference proceeding by Philippe~\cite{badillo2013group}, ISBI 2012 by César~\cite{gaudes2012structured}, and Farouj with a different formulation. Say conceptual differences among those.
        \item Mention stability-selection~\cite{meinshausen2010stability}
        \item Debiasing
        \item Connected to debiasing other deconvolution algorithms that are based on a norm lower than 1.
    \end{itemize}
\end{itemize}

\begin{table}
    \centering
    \begin{tabular}{ m{4cm} m{4cm} }
        Paradigm Free Mapping & Total Activation \\
        \toprule
        \begin{itemize}
            \pro 
        \end{itemize} &
        \begin{itemize}
            \pro Both the spike and block models solve the regularization problem with the same HRF.
            \con 
        \end{itemize} \\
    \end{tabular}
    \caption{Pros and cons of Paradigm Free Mapping and Total Activation.}
\end{table}

This work demonstrates that Paradigm Free Mapping and Total Activation yield practically identical results when the same HRF model and regularization parameter are employed, demonstrating that synthesis and analysis models are equivalent for temporal fMRI deconvolution. Thus, previously observed differences in performance must be due to differences in usage options. With the equivalence in the temporal deconvolution demonstrated, it can be assumed that additional regularization terms in the spatial or temporal domains would not modify this equivalence when convex operators are employed; e.g.\ when the regularization problem can be solved by means of the Fast Iterative Shrinkage-Thresholding Algorithm (FISTA)~\cite{beck2009FastIterativeShrinkagethresholding} or the Generalized Forward-Backward Splitting~\cite{raguet2013GeneralizedForwardBackwardSplittinga} techniques.

Future work should focus on investigating the appropriate formulation depending on data acquisition (e.g.\ single-echo vs multi-echo), accounting for HRF variability, robust methods to select the regularization parameter, and other potential \(\ell_{p,q}\)-norm regularization terms (e.g. \(p < 1\)) or debiasing approaches.