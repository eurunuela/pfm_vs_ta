\section{Discussion}

\begin{itemize}
    \item Pros and cons of each formulation: analysis vs synthesis
    \item Link with other approaches
    \item Finish with conclusions and a moving forward
    \begin{itemize}
        \item We have to refine the deconvolution
        \item HRF variability there are three: conference proceeding by Philippe~\cite{badillo2013group}, ISBI 2012 by César~\cite{gaudes2012structured}, and Farouj with a different formulation. Say conceptual differences among those.
        \item Mention stability-selection~\cite{meinshausen2010stability}
        \item Debiasing
        \item Connected to debiasing other deconvolution algorithms that are based on a norm lower than 1.
    \end{itemize}
\end{itemize}

% \begin{table}
%     \begin{tabularx}{\linewidth}{>{\parskip1ex}X@{\kern4\tabcolsep}>{\parskip1ex}X}
%     \toprule
%     \hfil\bfseries Pros
%     &
%     \hfil\bfseries Cons
%     \\\cmidrule(r{3\tabcolsep}){1-1}\cmidrule(l{-\tabcolsep}){2-2}
    
%     %% PROS, seperated by empty line or \par
%     It's nice\par
%     \lipsum[1]
%     It's very nice\par
    
%     &
    
%     %% CONS, seperated by empty line or \par
%     It's ugly\par
%     It's really ugly\par
%     \lipsum[4]
    
%     \\\bottomrule
%     \end{tabularx}
%     \caption{Pros and cons of Paradigm Free Mapping and Total Activation.}
% \end{table}

Paradigm Free Mapping and Total Activation yield practically identical performance when the same HRF model and regularization parameter are employed, demonstrating that synthesis and analysis models are equivalent for temporal fMRI deconvolution. Thus, previously observed differences in performance are due to usage options. Future work should focus on investigating the appropriate formulation depending on data acquisition (e.g. single-echo vs multi-echo), accounting for HRF variability, robust methods to select the regularization parameter, and other potential \(\ell_{p,q}\)-norm regularization terms (e.g. \(p < 1\)) or debiasing approaches.