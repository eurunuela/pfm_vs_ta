\section{Results}

\begin{itemize}
    \item Methods on how we're doing simulations and results (with simulations and experimental data)
    \begin{itemize}
        \item Different SNRs and maybe even use CAPs
        \item Selection of HRF explained if both use the same but it's different from what's used for simulating.
        \begin{itemize}
            \item What happens? For example with gamma for simulating.
        \end{itemize}
        \item Selection of regularization parameter
        \begin{itemize}
            \item Present with real data on a voxel
        \end{itemize}
    \end{itemize}
\end{itemize}

\hl{Figure 1} shows the difference in the hemodynamic response function that PFM and TA use by default; the SPMG1 and the HRF resulting from the linear differential operator respectively. While Paradigm Free Mapping allows for the use of any hemodynamic response function --- the columns of the design matrix \(\mathbf{H}\) are composed by shifted versions of the HRF --- the linear differential operator in TA is tailored for a fixed HRF. Hence, for practical reasons, we reproduced the HRF in the Total Activation filter and incorporated it into the PFM formulation.

\subsection{Selection of the regularization parameter based on the estimation of the noise}

\subsubsection{Simulated data}

\subsubsection{Experimental data}

\subsection{Selection of the regularization parameter by solving the regularization path}

\subsubsection{Simulated data}

\subsubsection{Experimental data}