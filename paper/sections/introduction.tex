% !TEX root = ../main.tex

\section{Introduction}

Functional magnetic resonance imaging (fMRI) data analysis is often directed to
identify and disentangle the neural processes that occur in different brain
regions during task or at rest. As the blood oxygenation level-dependent (BOLD)
signal of fMRI is only a proxy for neuronal activity mediated through
neurovascular coupling, an intermediate step that estimates the
activity-inducing signal, at the timescale of fMRI, from the BOLD timeseries can
be useful. Conventional analysis of task fMRI data relies on the general linear
models (GLM) to establish statistical parametric maps of brain activity by
regression of the empirical timecourses against hypothetical ones built from the
knowledge of the experimental paradigm. However, timing information of the
paradigm can be unknown, inaccurate, or insufficient in some scenarios such as
naturalistic stimuli, resting-state, or clinically-relevant assessments.

Deconvolution and methods alike are aiming to estimate neuronal activity by
undoing the blurring effect of the hemodynamic response, characterized as a
hemodynamic response function (HRF)\footnote{\textcolor{blue}{Note that the term
deconvolution is also alternatively employed to refer to the estimation of the
hemodynamic response shape assuming a known activity-inducing signal or neuronal
activity
\citep{Goutte2000Modelinghaemodynamicresponse,Marrelec2002Bayesianestimationhemodynamic,
Ciuciu2003Unsupervisedrobustnonparametric,Casanova2008impacttemporalregularization}.
}}. Given the inherently ill-posed nature of
hemodynamic deconvolution, due to the strong temporal low-pass characteristics
of the HRF, the key is to introduce additional constraints in the estimation
problem that are typically expressed as regularizers. For instance, the
so-called Wiener deconvolution is expressing a ``minimal energy'' constraint on
the deconvolved signal, and has been used in the framework of
psychophysiological interaction analysis to compute the interaction between a
seed's activity-inducing timecourse and an experimental modulation
\citep{Glover1999DeconvolutionImpulseResponse,Gitelman2003Modelingregionalpsychophysiologic,
Gerchen2014Analyzingtaskdependent,Di2018TaskConnectomicsExamining,
Freitas2020Timeresolvedeffective}.
Complementarily, the interest in deconvolution has increased to explore
time-varying activity in resting-state fMRI data
\citep{Preti2017dynamicfunctionalconnectome,Keilholz2017TimeResolvedResting,
Lurie2020Questionscontroversiesstudy,Bolton2020TappingMultiFaceted}.
In that case, the aim is to gain better insights of the neural signals that
drive functional connectivity at short time scales, as well as learning about
the spatio-temporal structure of functional components that dynamically
construct resting-state networks and their interactions
\citep{Karahanoglu2017Dynamicslargescale}.

Deconvolution of the resting-state fMRI signal has illustrated the significance
of transient, sparse spontaneous events
\citep{Petridou2012PeriodsrestfMRI,Allan2015FunctionalConnectivityMRI} that
refine the hierarchical clusterization of functional networks
\citep{Karahanoglu2013TotalactivationfMRI} and reveal their temporal overlap
based on their signal innovations not only in the human brain
\citep{Karahanoglu2015Transientbrainactivity}, but also in the spinal cord
\citep{Kinany2020DynamicFunctionalConnectivity}. Similar to task-related
studies, deconvolution allows to investigate modulatory interactions within and
between resting-state functional networks
\citep{Di2013ModulatoryInteractionsResting,Di2015Characterizationsrestingstate}.
In addition, decoding of the deconvolved spontaneous events allows to decipher
the flow of spontaneous thoughts and actions across different cognitive and
sensory domains while at rest
\citep{Karahanoglu2015Transientbrainactivity,GonzalezCastillo2019Imagingspontaneousflow,Tan_2017}.
Beyond findings on healthy subjects, deconvolution techniques have also proven
its utility in clinical conditions to characterize functional alterations of
patients with a progressive stage of multiple sclerosis at rest
\citep{Bommarito2020Alteredanteriordefault}, to find functional signatures
of prodromal psychotic symptoms and anxiety at rest on patients suffering from
schizophrenia \citep{Zoeller2019Largescalebrain}, to detect the foci of
interictal events in epilepsy patients without an EEG recording
\citep{Lopes2012Detectionepilepticactivity,Karahanoglu2013Spatialmappinginterictal},
or to study functional dissociations observed during non-rapid eye movement
sleep that are associated with reduced consolidation of information and impaired
consciousness \citep{Tarun2020NREMsleepstages}.

The algorithms for hemodynamic deconvolution can be classified based on the
assumed hemodynamic model and the optimization problem used to estimate the
neuronal-related signal. Most approaches assume a linear time-invariant model
for the hemodynamic response that is inverted by means of variational
(regularized) least squares estimators
\citep{Glover1999DeconvolutionImpulseResponse,Gitelman2003Modelingregionalpsychophysiologic,
Gaudes2010Detectioncharacterizationsingle,Gaudes2012Structuredsparsedeconvolution,
Gaudes2013Paradigmfreemapping,CaballeroGaudes2019deconvolutionalgorithmmulti,
HernandezGarcia2011Neuronaleventdetection,Karahanoglu2013TotalactivationfMRI,
Cherkaoui2019SparsitybasedBlind,
Huetel2021Hemodynamicmatrixfactorization,Costantini2022Anisotropic4DFiltering},
logistic functions
\citep{Bush2013Decodingneuralevents,Bush2015deconvolutionbasedapproach,
Loula2018DecodingfMRIactivity}, probabilistic mixture models
\citep{Pidnebesna2019EstimatingSparseNeuronal}, convolutional autoencoders
\citep{Huetel2018NeuralActivationEstimation} or nonparametric homomorphic
filtering \citep{Sreenivasan2015NonparametricHemodynamicDeconvolution}.
Alternatively, several methods have also been proposed to invert non-linear
models of the neuronal and hemodynamic coupling
\citep{Riera2004statespacemodel,Penny2005Bilineardynamicalsystems,Friston2008DEMvariationaltreatment,
Havlicek2011Dynamicmodelingneuronal,Aslan2016Jointstateparameter,
Madi2017HybridCubatureKalman,RuizEuler2018NonlinearDeconvolutionSampling}.

Among the variety of approaches, those based on regularized least squares
estimators have been employed more often due to their appropriate performance at
small spatial scales (e.g., voxelwise). Relevant for this work, two different
formulations can be established for the regularized least-squares deconvolution
problem, either based on a synthesis- or analysis-based model
\citep{Elad2007Analysisversussynthesis,ortelli2019synthesis}. The rationale
of the synthesis-based model is that we know or suspect that the true signal
(here, the neuronally-driven BOLD component of the fMRI signal) can be
represented as a linear combination of predefined patterns or dictionary atoms
(for instance, the hemodynamic response function). In contrast, the
analysis-based approach considers that the true signal is analyzed by some
relevant operator and the resulting signal is small (i.e., sparse).

As members of the groups that developed Paradigm Free Mapping (synthesis-based
\textcolor{blue}{solved with regularized least-squares estimators such as
ridge-regression \citealt{Gaudes2010Detectioncharacterizationsingle} or LASSO
\citealt{Gaudes2013Paradigmfreemapping}}) and Total Activation (analysis-based
\textcolor{blue}{also solved with a regularized least-squares estimator using
generalized total variation
\citealt{Karahanoglu2011SignalProcessingApproach,Karahanoglu2013TotalactivationfMRI}
}) deconvolution methods for fMRI data analysis, we are often contacted by
researchers who want to know about the similarities and differences between the
two methods and which one is better. \emph{It depends}---and to clarify this
point, this note revisits synthesis- and analysis-based deconvolution methods
for fMRI data and comprises four sections. First, we present the theory behind
these two deconvolution approaches based on regularized least squares estimators
that promote sparsity: Paradigm Free Mapping (PFM)
\citep{Gaudes2013Paradigmfreemapping} --- available in AFNI as
\textit{3dPFM}\footnote{\url{https://afni.nimh.nih.gov/pub/dist/doc/program_help/3dPFM.html}}
and
\textit{3dMEPFM}\footnote{\url{https://afni.nimh.nih.gov/pub/dist/doc/program_help/3dMEPFM.html}}
for single-echo and multi-echo data, respectively --- and Total Activation (TA)
\citep{Karahanoglu2013TotalactivationfMRI} --- available as part of the
\textit{iCAPs toolbox}\footnote{\url{https://c4science.ch/source/iCAPs/}}. We
describe the similarities and differences in their analytical formulations, and
how they can be related to each other. Next, we assess their performance
controlling for a fair comparison on simulated and experimental data. Finally,
we discuss their benefits and shortcomings and conclude with our vision on
potential extensions and developments.