\section{Introduction}

\begin{itemize}

    \item Talk about our motivation for this paper.

    \item We could mention iCAPs Neuron, and papers with applications like PFM, TA, clinical patient papers with iCAPs.

    \item Apart from [[Richard F. Betzel]]'s work~\cite{betzel2020temporal,esfahlani2020high,faskowitz2020edge}, we could mention the connection with the
    [[Multiplication of Temporal Derivatives]] method~\cite{shine2015estimation,shine2016dynamics}.

    \begin{itemize}
        \item These are basically calculating the derivative, which is the same as applying a high-pass filter and calculating the correlation.
    \end{itemize}

\end{itemize}

% Footnote example
% \fntext[myfootnote]{Since 1880.}

There is an increasing interest in methods that aim to recover the underlying neuronal activity from functional magnetic resonance imaging (fMRI) data with no prior information of the timing of the blood oxygenation level-dependent (BOLD) events. One of such techniques is deconvolution, which does not consider task-related stimulus functions or any other specific cause of the underlying neuronal activity. In other words, deconvolution methods are capable of blindly estimating the neuronal activity, which makes them especially attractive for exploring time-varying activity of resting-state fluctuations~\cite{petridou2013periods,karahanouglu2015transient,karahanouglu2017dynamics,kinany2020dynamic,gonzalez2019imaging}, naturalistic paradigms~\cite{betzel2020temporal}, or clinical conditions such as the study of interictal events in epilepsy.

Paradigm Free Mapping (PFM)~\cite{gaudes2013paradigm} --- which is available as \textit{3dPFM} in AFNI --- and Total Activation (TA)~\cite{karahanouglu2013total} are two of such deconvolution algorithms.

This note comprises three sections. In the first, we present the theory behind the Paradigm Free Mapping and Total Activation deconvolution algorithms. We then assess their performance using the same hemodynamic response function with different criteria for the selection of the regularization parameter: a) a selection based on the Bayesian (BIC)~\cite{schwarz1978estimating} and Akaike Information Criterion (AIC)~\cite{akaike1998information}, and b) a selection based on the estimated standard deviation of the noise in the data. We report that both methods produce identical results when estimating the underlying activity-inducing and innovation signals in different signal-to-noise ratio (SNR) settings. In the final section, we discuss the pros and cons of each of the described techniques and conclude with future steps.