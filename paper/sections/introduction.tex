\section{Introduction}

% \begin{itemize}

%     \item Talk about our motivation for this paper.

%     \item We could mention iCAPs Neuron, and papers with applications like PFM, TA, clinical patient papers with iCAPs.

%     \item Apart from [[Richard F. Betzel]]'s work~\citealt{betzel2020temporal,esfahlani2020high,faskowitz2020edge}, we could mention the connection with the
%     [[Multiplication of Temporal Derivatives]] method~\citealt{shine2015estimation,shine2016dynamics}.

%     \begin{itemize}
%         \item These are basically calculating the derivative, which is the same as applying a high-pass filter and calculating the correlation.
%     \end{itemize}

% \end{itemize}

% Footnote example
% \fntext[myfootnote]{Since 1880.}

Functional magnetic resonance imaging (fMRI) data analysis is often directed to disentangling and understanding the neural processes that occur among brain regions. While interactions in the brain are electrical in nature, the blood oxygenation level-dependent (BOLD) signal present in fMRI data reflects hemodynamics. Thus, an intermediate step that estimates the underlying neuronal activity from the BOLD signal can prove to be useful for understanding such interactions. Often, the analysis of task fMRI data relies on general linear models (GLM) to detect maps of brain activity by using the information about the timings of the BOLD events. However, this information can be unkown, innacurate, or insufficient in some scenarios. In such cases, and given the nature of the BOLD signal, the appropriate approximation of the neuronal activity can be obtained by means of deconvolution with an assumed hemodynamic response (\citealt{gitelman2003ModelingRegionalPsychophysiologic}).

Deconvolution and methods alike are becoming more popular for exploring time-varying activity in fMRI data within a number of neuroimaging studies due to their potential to blindly disentangle neural dynamics. One of such cases is the study of resting-state fluctuations with the aim of gaining insight into the origin of the signals driving functional connectivity and its temporal dynamics, as well as the organizational principles of brain function; i.e., to study and deconstruct the spatio-temporal structure of functional components that dynamically construct resting-state networks (\citealt{petridou2013PeriodsRestFMRI,karahanoglu2015TransientBrainActivity,karahanoglu2017DynamicsLargescaleFMRI,kinany2020DynamicFunctionalConnectivity, gonzalez-castillo2019ImagingSpontaneousFlow,allan2015FunctionalConnectivityMRI}). Deconvolution techniques can also prove to be helpful in clinical conditions to characterize functional alterations of patients with a progressive stage of multiple sclerosis at rest (\citealt{bommarito2020FunctionalNetworkDynamicsa}), to find functional signatures of prodromal psychotic symptoms and anxiety at rest on patients suffering from schizophrenia (\citealt{zoller2019LargeScaleBrainNetwork}), to detect the foci of interictal events in epilepsy patients without an EEG recording (\citealt{lopes2012DetectionEpilepticActivity}), or to study functional dissociations observed during non-rapid eye movement sleep that are associated with reduced consolidation of information and impaired consciousness (\citealt{tarun2021NREMSleepStagesa}).

A series of recent studies have also attempted to understand neural processes by studying the interactions between BOLD responses without estimating the underlying neuronal activity. For instance, co-activation patterns have been used to replicate seed correlation-based resting-state functional networks with a small portion of the data (\citealt{liu2013time,liu2013decomposition,liu2018co,majeed2009spatiotemporal,majeed2011spatiotemporal,cifre2020revisiting,cifre2020further,zhang2020relationship}). Likewise, the dynamics of functional connectivity have recently been investigated with the use of co-fluctuations and edge-centric techniques on tasks (\citealt{faskowitz2021EdgecentricModelHarmonizing}), resting-state (\citealt{zamaniesfahlani2020HighamplitudeCofluctuationsCortical}) and naturalistic paradigms (\citealt{faskowitz2020EdgecentricFunctionalNetwork,betzel2020TemporalFluctuationsBrain}). Methods based on the multiplication of temporal derivatives have also been presented for the estimation of dynamic functional connectivity on task fMRI data (\citealt{shine2015estimation,shine2016dynamics}).

This note revisits synthesis- and analysis-based deconvolution methods for fMRI data and comprises four sections. In the first, we present the theory behind two state-of-the-art deconvolution approaches based on estimators that promote sparsity: Paradigm Free Mapping (PFM) (\citealt{caballerogaudes2013ParadigmFreeMapping}) --- available as \textit{3dPFM} and \textit{3dMEPFM} in AFNI --- and Total Activation (TA) (\citealt{karahanoglu2013TotalActivationFMRI}) --- available as part of the \textit{iCAPs toolbox}. We then assess their performance controlling for a fair comparison on simulated and experimental data. Finally, we discuss the benefits and shortcomings of the techniques and conclude with our vision on potential extensions and developments.