\section{Theory}

\begin{itemize}
    \item What is deconvolution and different formulations presented as a review.
    \item Analysis vs synthesis
    \begin{itemize}
        \item TA paper but without the spatial regularization
        \item PFM paper
        \item In Gitelman it's an \(\mathbf{H}\) multiplied by a Fourier term.
    \end{itemize}
\end{itemize}

The hemodynamic response to neuronal activity \(y_t\) at time \(t\) can be modeled as the convolution with a finite impulse response function of the neuronal signal \(x_{t-\tau}\) at time \(t-\tau\) with the haemodynamic response function \(h_{\tau}\):

\begin{equation}
    y_t = \Sigma_t h_{\tau} x_{t-\tau}
\end{equation}

Functional MRI (fMRI) data analyses are often directed to disentangling and understanding the neural processes that occur among brain regions. However, interactions in the brain are expressed, not at the level of hemodynamic responses, but at the neural level. Thus, an intermediate step that estimates the underlying neuronal activity is necessary for such analyses. Given the nature of the fMRI BOLD signal, the appropriate approximation of the neuronal activity can be obtained by means of deconvolution with an assumed hemodynamic response\cite{gitelman2003modeling}.

\subsection{Paradigm Free Mapping}

\begin{equation}
    \hat{\mathbf{y}} = \arg \min_{\mathbf{y}} \| \mathbf{y} - \mathbf{Hs} \|_2^2 \quad \text{subject to} \quad \| \mathbf{s} \|_p \leq \alpha
\end{equation}

\subsection{Total Activation}