\documentclass[5p]{elsarticle}

\usepackage{lineno,hyperref}
\modulolinenumbers[5]

\journal{Neuroimage Technical Note}

%%%%%%%%%%%%%%%%%%%%%%%
%% Elsevier bibliography styles
%%%%%%%%%%%%%%%%%%%%%%%
%% To change the style, put a % in front of the second line of the current style and
%% remove the % from the second line of the style you would like to use.
%%%%%%%%%%%%%%%%%%%%%%%

%% Numbered
\bibliographystyle{model1-num-names}

%% Numbered without titles
%\bibliographystyle{model1a-num-names}

%% Harvard
%\bibliographystyle{model2-names.bst}\biboptions{authoryear}

%% Vancouver numbered
%\usepackage{numcompress}\bibliographystyle{model3-num-names}

%% Vancouver name/year
%\usepackage{numcompress}\bibliographystyle{model4-names}\biboptions{authoryear}

%% APA style
%\bibliographystyle{model5-names}\biboptions{authoryear}

%% AMA style
%\usepackage{numcompress}\bibliographystyle{model6-num-names}

%% `Elsevier LaTeX' style
% \bibliographystyle{elsarticle-num}
%%%%%%%%%%%%%%%%%%%%%%%

\begin{document}

\begin{frontmatter}

\title{Paradigm Free Mapping vs Total Activation}

%% Group authors per affiliation:
\author[bcbl,upv]{Eneko Uru\~nuela}
\author[bcbl]{C\'{e}sar Caballero-Gaudes}

\address[bcbl]{Basque Center on Cognition, Brain and Language, Spain}
\address[upv]{University of the Basque Country, Spain}

\begin{abstract}
Here's where the fantastic abstract will go.
\end{abstract}

\begin{keyword}
deconvolution, paradigm free mapping, total activation
\end{keyword}

\end{frontmatter}

\linenumbers

\section{Introduction}

\begin{itemize}

    \item Talk about our motivation for this paper.

    \item We could mention iCAPs Neuron, and papers with applications like PFM, TA, clinical patient papers with iCAPs.

    \item Apart from [[Richard F. Betzel]]'s work, we could mention the connection with the [[Multiplication of Temporal Derivatives]] method

    \begin{itemize}
        \item See \cite{shine2015estimation}
        \item See \cite{shine2016dynamics}
        \item These are basically calculating the derivative, which is the same as applying a high-pass filter and calculating the correlation.
    \end{itemize}

\end{itemize}

Here is a sample reference: \cite{gitelman2003modeling}.

\section{Theory}

\begin{itemize}
    \item What is deconvolution and different formulations presented as a review.
    \item Analysis vs synthesis
    \begin{itemize}
        \item TA paper but without the spatial regularization
        \item PFM paper
        \item In gitelman it's an $\mathbf{H}$ multiplied by a fourier term.
    \end{itemize}
\end{itemize}

\section{Results}

\begin{itemize}
    \item Methods on how we're doing simulations and results (with simulations and experimental data)
    \begin{itemize}
        \item Different SNRs and maybe even use CAPs
        \item Selection of HRF explained if both use the same but it's different from what's used for simulating.
        \begin{itemize}
            \item What happens? For example with gamma for simulating.
        \end{itemize}
        \item Selection of regularization parameter
        \begin{itemize}
            \item Present with real data on a voxel
        \end{itemize}
    \end{itemize}
\end{itemize}

\section{Discussion}

\section*{References}

\bibliography{mybibfile}

\end{document}